\documentclass[a4paper,10pt]{article}
\usepackage{paper-en}

\def\thetitle{Two Moons of Berk Ceylan}
\def\theauthors{}
%\def\theauthors{Anton Petrunin}

\hypersetup{colorlinks=true,
citecolor=black,
linkcolor=black,
anchorcolor=black,
filecolor=black,
menucolor=black,
urlcolor=black,
pdftitle={\thetitle},
pdfauthor={\theauthors}
}

%\overfullrule=100mm

\begin{document}


%\pagestyle{empty}\renewcommand\includegraphics[2][{}]{}

\title{\thetitle}
\author{}
\date{}
\maketitle

The following statement was conjectured by Berk Ceylan \cite{ceylan2023}.

\parit{Let $F$ be a plane figure bounded by a curve whose curvature lies between $\mp1$
Suppose the perimeter of $F$ is at least $4\cdot\pi$.
Then $F$ contains two unit discs with no common interior points.}

\medskip

The bound $4\cdot\pi$ is optimal; the most obvious borderline case is the disc of radius 2, but several other boarder cases are shown on the diagram; their boundary curves made from arcs with curevature $\pm1$ and $\tfrac13$.

\begin{figure}[ht!]
\vskip-0mm
\centering
\includegraphics{mppics/pic-10}
\vskip0mm
\end{figure}

The proof will be given as a sequence of exercises.

A disc in $F$ will be called an \emph{indisc} if it is maximal (with respect to inclusion) among discs of radius at most 1 in $F$;
in particular, any disc of radius 1 in $F$ is an indisc.

\begin{figure}[ht!]
\vskip-0mm
\centering
\includegraphics{mppics/pic-20}
\caption*{Indiscs.}
\vskip0mm
\end{figure}


\begin{enumerate}[(i)]

\item
Show that $F$ is the union of all its indiscs.

\item 
Assume $F$ is a counterexample, so its perimeter is at least $4\cdot\pi$, but it does not contain two nonoverlapping unit discs.
By Berk's result, the diameter of $F$ is less than $4$.
Apply approximation to show that we can assume that $F$ is bounded by a finite number, say $m$, of circle arcs with curvature $\pm1$ and has no corner points.

\item\label{incircle-intersect}
Follow the proof of Berk Ceylan \cite{ceylan2023} to show that any two indiscs in $F$ intersect.

\item\label{r+2}
Let $D\subset F$ be an indisc of radius $r$, and $D'$ be the concentric disc with radius $r+2$.
Show that $D'\supset F$.

\item
Denote by $H$ the intersection of all closed discs of radius $3$ that contain~$F$.
Use \ref{r+2} to show that $H$ is nonempty and its diameter is less than $4$.

\item
Apply Crofton's formula to prove that perimeter of $H$ is less than $4\cdot\pi$.

\item\label{alpha-beta}
The boundary of $H$ contains parts of boundary of $F$ and possibly several arcs, say $\alpha_1,\dots, \alpha_n$ of radius $3$.
Each arc $\alpha_i$ corresponds to an arc, say $\beta_i$ of $\partial F$.
Since the perimeter of $F$ is at least $4\cdot\pi$, we can choose index $i$ such that
\[\length\beta_i>\length\alpha_i;\]
set $\alpha=\alpha_i$ and $\beta=\beta_i$,


\begin{figure}[ht!]
\vskip-0mm
\centering
\includegraphics{mppics/pic-30}
\vskip0mm
\end{figure}

\item Let $\alpha'$ and $\beta'$ be
outer equidistant curves of $\alpha$ and $\beta$ at distance $1$.
Use \ref{incircle-intersect} to show that $\beta'$ does not have self-intersections.

\item\label{length-prime} Show that $\length\beta'-\length\beta=\length\alpha'-\length\alpha$.
Conclude that
\[\length\beta'>\length\alpha'.\]

\item\label{2-discs} Note that $\alpha'$ has curvature radius $4$ and $\beta'$ made from several arcs of curvature radius $2$.
Consider a circle $\sigma$ of radius $2$ that extends an arc of $\beta'$.
Show that (1) the center of arc $\alpha'$ lies inside or on $\sigma$ and (2) points of $\beta'$ lie outside or on $\sigma$.

\item Use the last statement to show that radial projection $\rho\:\beta'\to \alpha'$ with respect to the center of $\alpha'$ does not increase length. Arrive at a contradiction with~\ref{length-prime}.
\qeds

\end{enumerate}

{\sloppy
\def\emph{\textit}
\printbibliography[heading=bibintoc]
\fussy
}
\end{document}

\item Suppose $\beta$ contains at least 4 arcs.
Observe that $\beta_i$ is a curve in a family of the same type shown on the diagram;
these curves share the endpoints and the directions at the endpoints.
Show that the length of $\beta_i$ can be increased by moving it in the family.

\item Varying $\beta_i$ in this family we can get a figure $F'$ with larger perimeter that still lies in $H$ and meets one of the following conditions: $\partial F'$ kisses itself at some point or it kisses $\partial H$ at an additional point or the corresponding arc $\beta_i'$ in $F'$ has exactly 3 arcs.
Note that $F'$ is still a counterexample; indeed, it lies in the figure $H$ of diameter less than $4$.

\item
If $\beta_i'$ has exactly 3 arcs, then it looks like the the curve on the last diagram.
Observe that in this case $\beta_i$ and $\alpha_i$ have the same length and arrive at a contradiction.

\item
Suppose $\partial F'$ kisses itself, so it looks like one of two figures on the diagram.
In the first case, by the Moon in the puddle theorem, each of the parts of $F'$ subdivided by the tangency point contains a unit disc --- a contradiction.
In the second case, observe that two incircles of $F'$ at the tangency point do not intersect.
Therefore applying
