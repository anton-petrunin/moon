\documentclass{article}
\usepackage{moon}
\hypersetup{pdftitle={Moon in a puddle and four-vertex theorem},
pdfauthor={Anton Petrunin}}


\begin{document}
\title{Moon in a puddle and\\ the four-vertex theorem}
\author{Anton Petrunin and Sergio Zamora Barrera}
\date{}
\maketitle

The theorem about the Moon in a puddle by Vladimir Ionin and German Pestov \cite{pestov-ionin} is not very well known, but it deserves to be included in standard introductory texts on differential geometry of curves.
It gives the simplest meaningful example of a local-to-global theorem which what differential geometry is about.

This note is divided in two parts: we present a proof of the Moon in a puddle theorem and we use its key lemma to give a new proof of the four-vertex theorem. 


\section*{Moon in a puddle}

\begin{thm}{Theorem}\label{thm:moon-orginal}
Assume $\gamma$ is a simple closed smooth regular plane curve with curvature bounded by~1.
Then it surrounds a unit disc.
\end{thm}


{

\begin{wrapfigure}{r}{33 mm}
\vskip-6mm
\centering
\includegraphics{mppics/pic-62}
\vskip0mm
\end{wrapfigure}

A straightforward approach would be to start with some disc in the region bounded by the curve and blow it up to maximize its radius.
However, as one may see from the spinner-like example on the diagram it does not always lead to a solution --- a closed plane curve of curvature at most 1 may surround a disc of radius smaller than 1 that cannot be enlarged continuously.

}

We present the proof from \cite{petrunin-zamora} which is slight improvement of the original proof.
Another approach is sketched in~\cite{petrunin-2020,panov-petrunin}, and \cite[Problem 1.7.19]{toponogov}.

Let us use term \emph{circline} as a shortcut for \emph{circle or line}.
Note that an osculating circline of a smooth regular curve is defined at each of its points --- no need to assume that the curvature does not vanish.

We say that a circline $\sigma$ \emph{supports} a curve $\gamma$ at a point $p$ if the point $p$ lies on both $\sigma$ and $\gamma$ and the curve $\gamma$ lies in one of the closed regions that $\sigma$ cuts from the plane.

\begin{thm}{Key lemma}\label{thm:moon}
Assume $\gamma$ is a simple smooth regular plane loop.
Then at one point of $\gamma$ (distinct from its base) its osculating circle $\sigma$ globally supports $\gamma$ from the inside.
\end{thm}

First let us show that the theorem follows from the lemma.

\parit{Proof of \ref{thm:moon-orginal} modulo \ref{thm:moon}.}
Since $\gamma$ has absolute curvature at most 1, each osculating circle has radius at least 1.
According to the key lemma one of the osculating circles $\sigma$ globally supports $\gamma$ from inside.
In particular $\sigma$ lies inside of $\gamma$, whence the result.
\qeds

\parit{Proof of \ref{thm:moon}.}
Denote by $F$ the closed region surrounded by $\gamma$.
We can assume that $F$ lies on the left from $\gamma$.
Arguing by contradiction,
assume that the osculating circle at each point $p\in \gamma$ does not lie in~$F$.

\begin{figure}[!ht]%{r}{38 mm}
\vskip-0mm
\centering
\includegraphics{mppics/pic-32}
\vskip0mm
\end{figure}

Given a point $p\in\gamma$ let us consider the maximal circle $\sigma$ that lies completely in $F$ and tangent to $\gamma$ at $p$.
The circle $\sigma$ will be called the {}\emph{incircle} of $F$ at $p$.

Note that the curvature $\skur_\sigma$ of the incircle $\sigma$ has to be larger than the signed curvature of $\gamma$ at $p$, further denoted by $\skur(p)_\gamma$.
It follows that $\sigma$ has to touch $\gamma$ at another point.
Otherwise we can increase $\sigma$ slightly while keeping it inside $F$.

\begin{wrapfigure}{r}{32 mm}
\vskip-4mm
\centering
\includegraphics{mppics/pic-64}
\caption*{Two ovals pretend to be circles.}
\vskip0mm
\end{wrapfigure}

Choose a point $p_1$ on $\gamma$ that is distinct from its base point. 
Let $\sigma_1$ be the incircle at $p_1$.
Denote by $\gamma_1$ an arc of $\gamma$ from $p_1$ to a first point $q_1$ on $\sigma_1$.
Denote by $\hat\sigma_1$ and $\check\sigma_1$ two arcs of $\sigma_1$ from $p_1$ to $q_1$ such that the cyclic concatenation of $\hat\sigma_1$ and $\gamma_1$ surrounds~$\check\sigma_1$. 

Let $p_2$ be the midpoint of $\gamma_1$.
Denote by $\sigma_2$ the incircle at $p_2$.

Note that $\sigma_2$ cannot intersect $\hat\sigma_1$.
Otherwise, if $\sigma_2$ intersects $\hat\sigma_1$ at some point $s$, then $\sigma_2$ has to have two more common points with $\check\sigma_1$, say $x$ and $y$ --- one for each arc of $\sigma_2$ from $p_2$ to $s$.
Therefore $\sigma_1\z=\sigma_2$ since these two circles have three common points: $s$, $x$, and $y$. 
On the other hand, by construction, $p_2\in \sigma_2$ and $p_2\notin \sigma_1$ --- a contradiction.


Recall that $\sigma_2$ has to touch $\gamma$ at another point.
From above it follows that it can only touch $\gamma_1$ and therefore we can choose an arc $\gamma_2\subset \gamma_1$ that runs from $p_2$ to a first point $q_2$ on $\sigma_2$.
Since $p_2$ is the midpoint of $\gamma_1$, we have that
\[\length \gamma_2< \tfrac12\cdot\length\gamma_1.\eqlbl{eq:length<length/2}\]

Repeating this construction recursively,
we get an infinite sequence of arcs $\gamma_1\supset \gamma_2\supset\dots$;
by \ref{eq:length<length/2}, we also get that 
\[\length\gamma_n\to0\quad\text{as}\quad n\to\infty.\] 
Therefore the intersection 
\[\bigcap_n\gamma_n\]
contains a single point; denote it by $p_\infty$.

Let $\sigma_\infty$ be the incircle at $p_\infty$; it has to touch $\gamma$ at another point, say $q_\infty$.
The same argument as above shows that $q_\infty\in\gamma_n$ for any $n$.
It follows that $q_\infty =p_\infty$ --- a contradiction.
\qeds

\begin{thm}{Exercise}\label{ex:moon-rad}
Assume that a closed smooth regular curve $\gamma$ lies in a figure $F$ bounded by a closed simple plane curve.
Suppose that $R$ is the maximal radius of discs that lie in $F$.
Show that absolute curvature of $\gamma$ is at least $\tfrac1R$ at some parameter value.
\end{thm}


\section{Four-vertex theorem}
\index{Four-vertex theorem}

The key lemma in the previous section implies a generalization of the famous four-vertex theorem stated below.

Recall that a \emph{vertex} of a smooth regular curve is defined as a critical point of its signed curvature;
in particular, any local minimum (or maximum) of the signed curvature is a vertex.
For example, every point of a circle is its vertex.

The classical four-vertex theorem says that \emph{any closed smooth regular plane curve without self-interections have at least four vertexes}.
It has a number of different proofs and generalizations.
One of our favorite proofs was given by Robert Osserman \cite{osserman}; this paper also contains a short account on the history of the theorem.

Note that if an osculating circline $\sigma$ at a point $p$ supports $\gamma$, then $p$ is a vertex.
The latter is easy to check directly, but it also follows from Tait--Kneser spiral \cite{ghys-tabachnikov-timorin}.
Therefore the following theorem is indeed a generalization of the four-vertex theorem:

\begin{thm}{Theorem}\label{thm:4-vert}
Any smooth regular simple plane curve has is supported by its osculating circline at 4 distinct points; two from inside and two from outside.
\end{thm}

{

\begin{wrapfigure}{r}{33 mm}
\vskip-6mm
\centering
\includegraphics{mppics/pic-63}
\vskip0mm
\end{wrapfigure}

\parit{Proof.}
According to the key lemma (\ref{thm:moon}), there is a point $p\in\gamma$ such that its osculating circle supports $\gamma$ from inside.
The curve $\gamma$ can be considered as a loop with the base at $p$.
Therefore the key lemma implies the existence of another point $q\in\gamma$ with the same property.

It shows the existence of two osculating circles that support $\gamma$ from inside;
it remains to show existence of two osculating circles that support $\gamma$ from outside.

}

Let us apply to $\gamma$ an inversion with respect to a circle whose center lies inside~$\gamma$, then the obtained curve $\gamma_1$ also has  two osculating circles that support $\gamma_1$ from inside.
Note that these osculating circlines are inverses of the osculating circlines of $\gamma$.
Indeed, osculating circleline can be defined as a circline that has second order of contact with $\gamma$ at the point.
It remains to note that inversion does not change the order of contact between curves.

Note that the region lying inside of $\gamma$ is mapped to the region outside of $\gamma_1$ and the other way around.
Therefore these two new circlines correspond to the osculating circlines supporting $\gamma$ from outside.
\qeds

\begin{wrapfigure}[5]{r}{25 mm}
\vskip-7mm
\centering
\includegraphics{mppics/pic-65}
\vskip0mm
\end{wrapfigure}

\begin{thm}{Advanced exercise}\label{ex:curve-crosses-circle}
Suppose $\gamma$ is a closed simple smooth regular plane curve and $\sigma$ is a circle.
Assume $\gamma$ crosses $\sigma$ at the points $p_1,\dots,p_{2{\cdot} n}$ and these points appear in the same cycle order on $\gamma$ and on $\sigma$.
Show that $\gamma$ has at least $2\cdot n$ vertices.

Construct an example of a closed simple smooth regular plane curve $\gamma$ with only 4 vertices that crosses a given circle at arbitrarily many points. 
\end{thm}



\sloppy
\printbibliography[heading=bibintoc]
\fussy

\end{document}
